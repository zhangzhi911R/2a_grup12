\section{Arjun Yuda Firwanda}
\subsection{Soal 1}
Isi jawaban soal ke-1

Kalau mau dibikin paragrap \textbf{cukup enter aja}, tidak usah pakai \verb|par| dsb

%\subsection{Soal 2}
%Isi jawaban soal ke-2

%\subsection{Soal 3}
%Isi jawaban soal ke-3

\section{Dwi Yulianingsih}
\subsection{Soal 1}
Isi jawaban soal ke-1

Kalau mau dibikin paragrap \textbf{cukup enter aja}, tidak usah pakai \verb|par| dsb

%\subsection{Soal 2}
%Isi jawaban soal ke-2

%\subsection{Soal 3}
%Isi jawaban soal ke-3

\section{Harun Ar-Rasyid}
\subsection{Soal 1}
Isi jawaban soal ke-1

Kalau mau dibikin paragrap \textbf{cukup enter aja}, tidak usah pakai \verb|par| dsb

%\subsection{Soal 2}
%Isi jawaban soal ke-2

%\subsection{Soal 3}
%Isi jawaban soal ke-3

\section{Sri Rahayu}
\subsection{Soal 1}
Isi jawaban soal ke-1

Kalau mau dibikin paragrap \textbf{cukup enter aja}, tidak usah pakai \verb|par| dsb

%\subsection{Soal 2}
%Isi jawaban soal ke-2

%\subsection{Soal 3}
%Isi jawaban soal ke-3

\section{Doli Jonviter}
\subsection{Soal 1}
Isi jawaban soal ke-1

Kalau mau dibikin paragrap \textbf{cukup enter aja}, tidak usah pakai \verb|par| dsb

%\subsection{Soal 2}
%Isi jawaban soal ke-2

%\subsection{Soal 3}
%Isi jawaban soal ke-3

\section{Rahmatul Ridha}
\subsubsection{Ketrampilan Pemrograman}
\begin{enumerate}
    \item Buatlah fungsi dengan inputan variabel NPM, dan melakukan print luaran huruf yang dirangkai dari tanda bintang, pagar atau plus dari NPM kita. Tanda bintang untuk NPM mod 3=0, tanda pagar untuk NPM mod 3 =1, tanda plus untuk NPM mod3=2.
    \lstinputlisting[firstline=69, lastline=103]{src/3/1144124/1144124Chapter3.py}

    \item Buatlah fungsi dengan inputan variabel berupa NPM. kemudian dengan menggunakan perulangan mengeluarkan print output sebanyak dua dijit belakang NPM.
    \lstinputlisting[firstline=105, lastline=112]{src/3/1144124/1144124Chapter3.py}

    \item Buatlah fungsi dengan dengan input variabel string bernama NPM dan beri luaran output dengan perulangan berupa tiga karakter belakang dari NPM sebanyak penjumlahan tiga dijit tersebut.
    \lstinputlisting[firstline=114, lastline=123]{src/3/1144124/1144124Chapter3.py}

    \item Buatlah fungsi hello word dengan input variabel string bernama NPM dan beri luaran output berupa digit ketiga dari belakang dari variabel NPM menggunakan akses langsung manipulasi string pada baris ketiga dari variabel NPM.
    \lstinputlisting[firstline=125, lastline=131]{src/3/1144124/1144124Chapter3.py}

    \item buat fungsi program dengan input variabel NPM dan melakukan print nomor npm satu persatu kebawah.
    \lstinputlisting[firstline=133, lastline=138]{src/3/1144124/1144124Chapter3.py}

    \item Buatlah fungsi dengan inputan variabel NPM, didalamnya melakukan penjumlahan dari seluruh dijit NPM tersebut, wajib menggunakan perulangan dan atau kondisi.
        \lstinputlisting[firstline=140, lastline=147]{src/3/1144124/1144124Chapter3.py}

    \item Buatlah fungsi dengan inputan variabel NPM, didalamnya melakukan melakukan perkalian dari seluruh dijit NPM tersebut, wajib menggunakan perulangan dan atau kondisi.
        \lstinputlisting[firstline=149, lastline=156]{src/3/1144124/1144124Chapter3.py}

    \item Buatlah fungsi dengan inputan variabel NPM, Lakukan print NPM anda tapi hanya dijit genap saja. wajib menggunakan perulangan dan atau kondisi.
        \lstinputlisting[firstline=257, lastline=263]{src/3/1144124/1144124Chapter3.py}

    \item Buatlah fungsi dengan inputan variabel NPM, Lakukan print NPM anda tapi hanya dijit ganjil saja. wajib menggunakan perulangan dan atau kondisi.
        \lstinputlisting[firstline=158, lastline=164]{src/3/1144124/1144124Chapter3.py}

    \item Buatlah fungsi dengan inputan variabel NPM, Lakukan print NPM anda tapi hanya dikit yang termasuk bilangan prima saja. wajib menggunakan perulangan dan atau kondisi.
        \lstinputlisting[firstline=166, lastline=172]{src/3/1144124/1144124Chapter3.py}

    \item Buatlah satu library yang berisi fungsi-fungsi dari nomor diatas dengan nama file epi.py dan berikan contoh cara pemanggilannya pada file main.py.
        \lstinputlisting[firstline=8, lastline=21]{src/3/1144124/main.py}

    \item Buatlah satu library class dengan nama gile kelas3lib.py yang merupakan modifikasi dari fungsi-fungsi nomor diatas dan berikan contoh cara pemanggilannya pada file mainn.py.
        \lstinputlisting[firstline=23, lastline=38]{src/3/1144124/main.py}

\end{enumerate}

\subsubsection{Ketrampilan Penanganan Error}
Error yang di dapat dari mengerjakan tugas ini adalah type error, cara menaggulaginya dengan cara mengecheck kembali codingannya
kemudian run kembali aplikasinya. Berikut contoh Penggunaan fungsi try dan exception :
\lstinputlisting[firstline=174, lastline=188]{src/3/1144124/1144124Chapter3.py}

\section{Tomy Prawoto}
\subsection{Soal 1}
Isi jawaban soal ke-1

Kalau mau dibikin paragrap \textbf{cukup enter aja}, tidak usah pakai \verb|par| dsb

%\subsection{Soal 2}
%Isi jawaban soal ke-2

%\subsection{Soal 3}
%Isi jawaban soal ke-3
