\section{Arjun Yuda Firwanda}
\subsection{Soal 1}
Buatlah fungsi (file terpisah/library dengan nama NPM csv.py) untuk membuka file csv dengan lib csv mode list
\lstinputlisting[firstline=8, lastline=21]{src/4/1174008/praktek/a1174008_csv.py}

\subsection{Soal 2}
Buatlah fungsi (file terpisah/library dengan nama NPM csv.py) untuk membuka file csv dengan lib csv mode dictionary
\lstinputlisting[firstline=23,lastline=55]{src/4/1174008/praktek/a1174008_csv.py}

\subsection{Soal 3}
Buatlah fungsi (file terpisah/library dengan nama NPM pandas.py) untuk membuka file csv dengan lib pandas mode list
\lstinputlisting[firstline=9,lastline=11]{src/4/1174008/praktek/a1174008_pandas.py}

\subsection{Soal 4}
Buatlah fungsi (file terpisah/library dengan nama NPM pandas.py) untuk membuka file csv dengan lib pandas mode dictionary
\lstinputlisting[firstline=13,lastline=16]{src/4/1174008/praktek/a1174008_pandas.py}

\subsection{Soal 5}
Buat fungsi baru di NPM pandas.py untuk mengubah format tanggal menjadi standar dataframe
\lstinputlisting[firstline=18,lastline=20]{src/4/1174008/praktek/a1174008_pandas.py}

\subsection{Soal 6}
Buat fungsi baru di NPM pandas.py untuk mengubah index kolom
\lstinputlisting[firstline=22,lastline=24]{src/4/1174008/praktek/a1174008_pandas.py}

\subsection{Soal 7}
Buat fungsi baru di NPM pandas.py untuk mengubah atribut atau nama kolom
\lstinputlisting[firstline=26,lastline=42]{src/4/1174008/praktek/a1174008_pandas.py}

\subsection{Soal 8}
Buat program main.py yang menggunakan library NPM csv.py yang membuat dan membaca file csv
\lstinputlisting[firstline=8,lastline=10]{src/4/1174008/praktek/main_arjun.py}

\subsection{Soal 9}
Buat program main2.py yang menggunakan library NPM pandas.py yang membuat dan membaca file csv
\lstinputlisting[firstline=12,lastline=14]{src/4/1174008/praktek/main_arjun.py}

\subsection{Penanganan Error}
Dalam praktek kali ini belum menemukan error

\section{Dwi Yulianingsih}
\subsection{Soal 1}
Buatlah fungsi (file terpisah/library dengan nama NPM csv.py) untuk membuka file csv dengan lib csv mode list
\lstinputlisting[firstline=10, lastline=20]{src/4/1174009/praktek/d1174009_csv.py}


\subsection{Soal 2}
Buatlah fungsi (file terpisah/library dengan nama NPM csv.py) untuk membuka file csv dengan lib csv mode dictionary
\lstinputlisting[firstline=22,lastline=34]{src/4/1174009/praktek/d1174009_csv.py}

\subsection{Soal 3}
Buatlah fungsi (file terpisah/library dengan nama NPM pandas.py) untuk membuka file csv dengan lib pandas mode list
\lstinputlisting[firstline=7,lastline=10]{src/4/1174009/praktek/d1174009_pandas.py}

\subsection{Soal 4}
Buatlah fungsi (file terpisah/library dengan nama NPM pandas.py) untuk membuka file csv dengan lib pandas mode dictionary
\lstinputlisting[firstline=12,lastline=15]{src/4/1174009/praktek/d1174009_pandas.py}

\subsection{Soal 5}
Buat fungsi baru di NPM pandas.py untuk mengubah format tanggal menjadi standar dataframe
\lstinputlisting[firstline=17,lastline=19]{src/4/1174009/praktek/d1174009_pandas.py}

\subsection{Soal 6}
Buat fungsi baru di NPM pandas.py untuk mengubah index kolom
\lstinputlisting[firstline=21,lastline=23]{src/4/1174009/praktek/d1174009_pandas.py}

\subsection{Soal 7}
Buat fungsi baru di NPM pandas.py untuk mengubah atribut atau nama kolom
\lstinputlisting[firstline=25,lastline=29]{src/4/1174009/praktek/d1174009_pandas.py}

\subsection{Soal 8}
Buat program main.py yang menggunakan library NPM csv.py yang membuat dan membaca file csv
\lstinputlisting[firstline=8,lastline=10]{src/4/1174009/praktek/main_dwi.py}

\subsection{Soal 9}
Buat program main2.py yang menggunakan library NPM pandas.py yang membuat dan membaca file csv
\lstinputlisting[firstline=12,lastline=14]{src/4/1174009/praktek/main_dwi.py}

\subsection{Penanganan eror}
Ada kalanya saat kita baca file, tapi filenya belum ada. Maka biasanya akan terjadi IOerror.
\lstinputlisting[firstline=8,lastline=8]{src/4/1174009/praktek/eror.py}
maka di tangani dengan cara seperti dibawah ini :
\lstinputlisting[firstline=10,lastline=13]{src/4/1174009/praktek/eror.py}
maka akan muncul peringatan seperti dibawah :
\lstinputlisting[firstline=15,lastline=15]{src/4/1174009/praktek/eror.py}



\section{Harun Ar-Rasyid}
\subsection{Soal 1}
Berikut adalah pemanggilan file csv dengan library csv yang menggunakan list
\lstinputlisting[firstline=10, lastline=20]{src/4/1174027/praktek/c_1174027_csv.py}

\subsection{Soal 2}
Berikut adalah pemanggilan file csv dengan library csv yang menggunakan dictionary
\lstinputlisting[firstline=22, lastline=31]{src/4/1174027/praktek/c_1174027_csv.py}

\subsection{Soal 3}
Berikut adalah pemanggilan file csv dengan library pandas yang menggunakan list
\lstinputlisting[firstline=9, lastline=11]{src/4/1174027/praktek/p_1174027_pandas.py}

\subsection{Soal 4}
Berikut adalah pemanggilan file csv dengan library pandas yang menggunakan dictionary
\lstinputlisting[firstline=13, lastline=16]{src/4/1174027/praktek/p_1174027_pandas.py}

\subsection{Soal 5}
Berikut penggunaan untuk merubah standar penulisan tanggal, yang mengikuti standar penulisan dari pandas.
\lstinputlisting[firstline=18, lastline=20]{src/4/1174027/praktek/p_1174027_pandas.py}

\subsection{Soal 6}
Berikut merupakan pergantian index kolom
\lstinputlisting[firstline=22, lastline=24]{src/4/1174027/praktek/p_1174027_pandas.py}

\subsection{Soal 7}
berikut merupakan penggunaan untuk merename atribut yang digunakan, atau merubah nama header 0
\lstinputlisting[firstline=26, lastline=30]{src/4/1174027/praktek/p_1174027_pandas.py}

\subsection{Soal 8}
\lstinputlisting[firstline=8, lastline=10]{src/4/1174027/praktek/main_harun.py}

\subsection{Soal 9}
\lstinputlisting[firstline=11, lastline=14]{src/4/1174027/praktek/main_harun.py}

\subsection{Penanganan Error}
Dalam praktek kali ini alhamdulliha tidak menemukan error

\section{Sri Rahayu}
\subsection{Soal 1}
Isi jawaban soal ke-1

Kalau mau dibikin paragrap \textbf{cukup enter aja}, tidak usah pakai \verb|par| dsb

%\subsection{Soal 2}
%Isi jawaban soal ke-2

%\subsection{Soal 3}
%Isi jawaban soal ke-3

\section{Doli Jonviter}
\subsection{Soal 1}
\begin{itemize}
\item Berikut adalah pemanggilan file csv dengan library csv yang menggunakan list
\lstinputlisting[firstline=10, lastline=20]{src/4/1154016/Chapter4/c_1154016_csv.py}

\item Berikut adalah pemanggilan file csv dengan library csv yang menggunakan dictionary
\lstinputlisting[firstline=22, lastline=31]{src/4/1154016/Chapter4/c_1154016_csv.py}

\item Berikut adalah pemanggilan file csv dengan library pandas yang menggunakan list
\lstinputlisting[firstline=9, lastline=11]{src/4/1154016/Chapter4/p_1154016_pandas.py}

\item Berikut adalah pemanggilan file csv dengan library pandas yang menggunakan dictionary
\lstinputlisting[firstline=13, lastline=16]{src/4/1154016/Chapter4/p_1154016_pandas.py}

\item Berikut penggunaan untuk merubah standar penulisan tanggal, yang mengikuti standar penulisan dari pandas.
\lstinputlisting[firstline=18, lastline=20]{src/4/1154016/Chapter4/p_1154016_pandas.py}

\item Berikut merupakan pergantian index kolom
\lstinputlisting[firstline=22, lastline=24]{src/4/1154016/Chapter4/p_1154016_pandas.py}


\item berikut merupakan penggunaan untuk merename atribut yang digunakan, atau merubah nama header 0
\lstinputlisting[firstline=26, lastline=30]{src/4/1154016/Chapter4/p_1154016_pandas.py}
\end{itemize}
\subsection{Soal 8}
\lstinputlisting[firstline=8, lastline=10]{src/4/1154016/Chapter4/jonviter.py}

\subsection{Soal 9}
\lstinputlisting[firstline=11, lastline=14]{src/4/1154016/Chapter4/jonviter.py}




%\subsection{Soal 2}
%Isi jawaban soal ke-2

%\subsection{Soal 3}
%Isi jawaban soal ke-3

\section{Rahmatul Ridha}
\subsection{Keterampilan Pemrograman}
\begin{enumerate}
	\item Buatlah  fungsi  (file  terpisah/library  dengan  nama  NPMcsv.py)  untuk  membuka file csv dengan lib csv mode list.
	
	\lstinputlisting[caption = Fungsi untuk membuka file CSV dengan lib CSV mode list., firstline=10, lastline=15]{src/4/1144124/Chapter4/1144124csv.py}
	
	\item Buatlah  fungsi  (file  terpisah/library  dengan  nama  NPMcsv.py)  untuk  membuka file csv dengan lib csv mode dictionary.
	
	\lstinputlisting[caption =  Fungsi untuk membuka file CSV dengan lib CSV mode dictionary., firstline=17, lastline=22]{src/4/1144124/Chapter4/1144124csv.py}
	
	\item Buatlah fungsi (file terpisah/library dengan nama NPMpandas.py) untuk membuka file csv dengan lib pandas mode list.
	
	\lstinputlisting[caption =  Fungsi untuk membuka file CSV dengan lib Pandas mode list., firstline=10, lastline=13]{src/4/1144124/Chapter4/1144124pandas.py}
	
	\item Buatlah fungsi (file terpisah/library dengan nama NPMpandas.py) untuk membuka file csv dengan lib pandas mode dictionary.
	
	\lstinputlisting[caption =  Fungsi untuk membuka file CSV dengan lib Pandas mode dictionary., firstline=10, lastline=13]{src/4/1144124/Chapter4/1144124pandas.py}
	
	\item  Buat fungsi baru di NPMpandas.py untuk mengubah format tanggal menjadi standar dataframe.
	
	\lstinputlisting[caption =  Fungsi untuk mengubah format tanggal menjadi standar dataframe., firstline=15, lastline=19]{src/4/1144124/Chapter4/1144124pandas.py}
	
	\item Buat fungsi baru di NPMpandas.py untuk mengubah index kolom.
	
	\lstinputlisting[caption =  Fungsi untuk mengubah index kolom., firstline=21, lastline=24]{src/4/1144124/Chapter4/1144124pandas.py}
	
	\item Buat fungsi baru di NPMpandas.py untuk mengubah atribut atau nama kolom.
	
	\lstinputlisting[caption =  Fungsi untuk mengubah atribut atau nama kolom., firstline=26, lastline=30]{src/4/1144124/Chapter4/1144124pandas.py}
	
	\item Buat program main.py yang menggunakan library NPMcsv.py yang membuat dan membaca file csv.
	
	\lstinputlisting[caption =  Membuat dan mebaca file CSV menggunakan library 1144124pandas., firstline=8, lastline=13]{src/4/1144124/Chapter4/main.py}
	
	\item Buat program main2.py yang menggunakan library NPMpandas.py yang membuat dan membaca file csv.
	
	\lstinputlisting[caption = Membuat dan mmebaca file CSV menggunakan library 1144124pandas., firstline=8, lastline=13]{src/4/1144124/Chapter4/main2.py}	
\end{enumerate}

\subsection{Penanganan eror}
Saat akan membaca file, akan tetapi file yang akan dibaca belum ada. Maka biasanya akan terjadi IOerror.
\lstinputlisting[firstline=8,lastline=8]{src/4/1144124/Chapter4/error.py}
maka penanganannya dengan cara seperti dibawah ini :
\lstinputlisting[firstline=10,lastline=13]{src/4/1144124/Chapter4/error.py}
dan akan muncul peringatan seperti dibawah :
\lstinputlisting[firstline=15,lastline=15]{src/4/1144124/Chapter4/error.py}

\section{Tomy Prawoto}
\subsection{Soal Praktek}
\begin{enumerate}

\item Buatlah fungsi (file terpisah/library dengan nama NPM csv.py) untuk membuka file csv dengan lib csv mode list
%\lstinputlisting[firstline=8, lastline=21]{src/4/1154121/chapter4/1154121_csv.py}

\item Buatlah fungsi (file terpisah/library dengan nama NPM csv.py) untuk membuka file csv dengan lib csv mode dictionary
%\lstinputlisting[firstline=22,lastline=33]{src/4/1154121/chapter4/1154121_csv.py}

\item Buatlah fungsi (file terpisah/library dengan nama NPM pandas.py) untuk membuka file csv dengan lib pandas mode list
%\lstinputlisting[firstline=9,lastline=12]{src/4/1154121/chapter4/1154121_pandas.py}

\item Buatlah fungsi (file terpisah/library dengan nama NPM pandas.py) untuk membuka file csv dengan lib pandas mode dictionary
%\lstinputlisting[firstline=14,lastline=21]{src/4/1154121/chapter4/1154121_pandas.py}

\item Buat fungsi baru di NPM pandas.py untuk mengubah format tanggal menjadi standar dataframe
%\lstinputlisting[firstline=22,lastline=25]{src/4/1154121/chapter4/1154121pandas.py}

\item Buat fungsi baru di NPM pandas.py untuk mengubah index kolom
%\lstinputlisting[firstline=27,lastline=31]{src/4/1154121/chapter4/1154121pandas.py}

\item Buat fungsi baru di NPM pandas.py untuk mengubah atribut atau nama kolom
%\lstinputlisting[firstline=33,lastline=37]{src/4/1154121/chapter4/1154121pandas.py}

\item Buat program main.py yang menggunakan library NPM csv.py yang membuat dan membaca file csv
%\lstinputlisting[firstline=8,lastline=13]{src/4/1154121/chapter4/main.py}

\item Buat program main2.py yang menggunakan library NPM pandas.py yang membuat dan membaca file csv
%\lstinputlisting[firstline=8, lastline=13]{src/4/1154121/chapter4/main2.py}
\end{enumerate}
