\section{Arjun Yuda Firwanda}
\subsection{Soal 1}
Isi jawaban soal ke-1

Kalau mau dibikin paragrap \textbf{cukup enter aja}, tidak usah pakai \verb|par| dsb

%\subsection{Soal 2}
%Isi jawaban soal ke-2

%\subsection{Soal 3}
%Isi jawaban soal ke-3

\section{Dwi Yulianingsih}
\subsection{Soal 1}
Isi jawaban soal ke-1

Kalau mau dibikin paragrap \textbf{cukup enter aja}, tidak usah pakai \verb|par| dsb

%\subsection{Soal 2}
%Isi jawaban soal ke-2

%\subsection{Soal 3}
%Isi jawaban soal ke-3

\section{Harun Ar-Rasyid}
\subsection{Soal 1}
Isi jawaban soal ke-1

Kalau mau dibikin paragrap \textbf{cukup enter aja}, tidak usah pakai \verb|par| dsb

%\subsection{Soal 2}
%Isi jawaban soal ke-2

%\subsection{Soal 3}
%Isi jawaban soal ke-3

\section{Sri Rahayu}
\subsection{Soal 1}
Isi jawaban soal ke-1

Kalau mau dibikin paragrap \textbf{cukup enter aja}, tidak usah pakai \verb|par| dsb

%\subsection{Soal 2}
%Isi jawaban soal ke-2

%\subsection{Soal 3}
%Isi jawaban soal ke-3

\section{Doli Jonviter}
\subsection{Soal 1}
Isi jawaban soal ke-1

Kalau mau dibikin paragrap \textbf{cukup enter aja}, tidak usah pakai \verb|par| dsb

%\subsection{Soal 2}
%Isi jawaban soal ke-2

%\subsection{Soal 3}
%Isi jawaban soal ke-3

\section{Rahmatul Ridha}
\subsubsection{Pemahanan Teori}
\begin{enumerate}
\item Apa itu fungsi, inputan fungsi dan kembalian fungsi dengan contoh kode program
    lainnya.
    \subitem Fungsi Fungsi adalah bagian dari program yang dapat digunakan ulang. Hal ini bisa dicapai dengan memberi nama pada blok statemen, kemudian nama ini dapat dipanggil di manapun dalam program. Kita telah menggunakan beberapa fungsi builtin seperti range. Fungsi dalam Python didefinisikan menggunakan kata kunci def. Setelah def ada nama pengenal fungsi diikut dengan parameter yang diapit oleh tanda kurung dan diakhir dingan tanda titik dua :. Baris berikutnya berupa blok fungsi yang akan dijalankan jika fungsi dipanggil.
        \lstinputlisting[firstline=8, lastline=9]{src/3/1144124/1144124Chapter3.py}

    \subitem Fungsi dapat membaca parameter, parameter adalah nilai yang disediakan kepada fungsi, dimana nilai ini akan menentukan output yang akan dihasilkan fungsi.
        \lstinputlisting[firstline=11, lastline=14]{src/3/1144124/1144124Chapter3.py}

     \subitem Statemen return digunakan untuk keluar dari fungsi. Kita juga dapat menspesifikasikan nilai kembalian.
        \lstinputlisting[firstline=16, lastline=25]{src/3/1144124/1144124Chapter3.py}

\item Apa itu paket dan cara pemanggilan paket atau library dengan contoh kode
    program lainnya.
    \subitem Untuk memudahkan dalam pemanggilan fungsi yang di butuhkan, agar dapat dipanggil berulang.
    Cara pemanggilannya
    \lstinputlisting[firstline=27, lastline=40]{src/3/1144124/1144124Chapter3.py}

\item Jelaskan Apa itu kelas, apa itu objek, apa itu atribut, apa itu method dan contoh kode program lainnya masing-masing.
    kelas merupakan sebuah blueprint yang mepresentasikan objek.
    objek adalah hasil cetakan dadri sebuah kelas.
    method adalah suatu upaya yang digunakan oleh object.
    \lstinputlisting[firstline=42, lastline=43]{src/3/1144124/1144124Chapter3.py}

\item Jelaskan cara pemanggikan library kelas dari instansiasi dan pemakaiannya dengan contoh program lainnya.
    Cara Pemanggilanya :
    \subitem pertama import terlebih dahulu filenya.
    \subitem kemudian buat variabel untuk menampung datanya.
    \subitem setelah itu panggil nama classnya dan panggil methodnya.
    \subitem Gunakan perintah print untuk menampilkan hasilnya.
      \lstinputlisting[firstline=46, lastline=52]{src/3/1144124/1144124Chapter3.py}

\item Jelaskan dengan contoh pemakaian paket dengan perintah from kalkulator im-port Penambahan disertai dengan contoh kode lainnya.
    Penggunaan paket from namafile import, itu berfungsi untuk memanggil file dan fungsinya :
    \lstinputlisting[firstline=54, lastline=56]{src/3/1144124/1144124Chapter3.py}

\item Jelaskan dengan contoh kodenya, pemakaian paket fungsi apabila file library ada di dalam folder.
    Pemakaian paket adalah perkumpulan fungsi-fungsi. contoh kodenya adalah sebagai berikut :

\item Jelaskan dengan contoh kodenya, pemakaian paket kelas apabila file library ada di dalam folder.
    \lstinputlisting[firstline=61, lastline=66]{src/3/1144124/1144124Chapter3.py}

\end{enumerate}

\section{Tomy Prawoto}
\subsection{Soal 1}
Isi jawaban soal ke-1

Kalau mau dibikin paragrap \textbf{cukup enter aja}, tidak usah pakai \verb|par| dsb

%\subsection{Soal 2}
%Isi jawaban soal ke-2

%\subsection{Soal 3}
%Isi jawaban soal ke-3
