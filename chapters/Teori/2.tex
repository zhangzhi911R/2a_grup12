\section{Arjun Yuda Firwanda}
\subsection{Soal 1}
Isi jawaban soal ke-1

Kalau mau dibikin paragrap \textbf{cukup enter aja}, tidak usah pakai \verb|par| dsb

%\subsection{Soal 2}
%Isi jawaban soal ke-2

%\subsection{Soal 3}
%Isi jawaban soal ke-3

\section{Dwi Yulianingsih}
\subsection{Soal 1}
Isi jawaban soal ke-1

Kalau mau dibikin paragrap \textbf{cukup enter aja}, tidak usah pakai \verb|par| dsb

%\subsection{Soal 2}
%Isi jawaban soal ke-2

%\subsection{Soal 3}
%Isi jawaban soal ke-3

\section{Harun Ar-Rasyid}
\subsection{Soal 1}
Isi jawaban soal ke-1

Kalau mau dibikin paragrap \textbf{cukup enter aja}, tidak usah pakai \verb|par| dsb

%\subsection{Soal 2}
%Isi jawaban soal ke-2

%\subsection{Soal 3}
%Isi jawaban soal ke-3

\section{Sri Rahayu}
\subsection{Soal 1}
Isi jawaban soal ke-1

Kalau mau dibikin paragrap \textbf{cukup enter aja}, tidak usah pakai \verb|par| dsb

%\subsection{Soal 2}
%Isi jawaban soal ke-2

%\subsection{Soal 3}
%Isi jawaban soal ke-3

\section{Doli Jonviter}
\subsection{Soal 1}
Isi jawaban soal ke-1

Kalau mau dibikin paragrap \textbf{cukup enter aja}, tidak usah pakai \verb|par| dsb

%\subsection{Soal 2}
%Isi jawaban soal ke-2

%\subsection{Soal 3}
%Isi jawaban soal ke-3

\section{Rahmatul Ridha}
\subsection{Teori}
\begin{enumerate}
\item jenis-jenis variable phyton dan cara pemakaiannya. Variable merupakan tempat untuk menyimpan data, Isi dari variabel itu dapat berubah atau mutable sesuai dengan operasi yang diinginkan. Saat program dieksekusi maka variabellah yang bertugas menyimpan data. Dimana didalam phyton terdapat beberapa variable diantaranya number, boolean,string. Dalam membuat variabel Pythoncaranya adalah sebagai berikut
    \lstinputlisting[firstline=8,lastline=46]{src/2/1144124/1144124_Teori.py}

\item operator dasar aritmatika. Dimana terdapat penjumlahan,pengurangan,pembagian,perkalian,perpangkatan,pembulatan nominal
    \lstinputlisting[firstline=47,lastline=75]{src/2/1144124/1144124_Teori.py}

\item Perulangan. Dalam phyton terdapat perulangan while dan for :
    \lstinputlisting[firstline=76,lastline=87]{src/2/1144124/1144124_Teori.py}

\item Dimana terdapat sintak untuk memilih kondisi didalam kondisi. Untuk memilih keputusan menggunakan (kondisi if) dimana digunakan untuk mengantisipasi kondisi yang terjadi saat jalannya suatu program dan menentukan tindakan apa yang akan dilakukan sesuai dengan kondisi.
    \lstinputlisting[firstline=88,lastline=112]{src/2/1144124/1144124_Teori.py}

\item Jenis-jenis sintak error pada phyton.

Syntax errors Jika dalam program terdapat kesalahan sintaks maka proses akan berhenti dan menampilkan pesan kesalahan.
Runtime errors, disebut begitu karenakesalahan tidak akan muncul sampai Anda menjalankan program tersebut.Kesalahan ini juga dikenal dengan exceptions atau pengecualian karena biasanya mengindikasikan sesuatu pengecualian yang buruk telah terjadi.

Type eror merupakan eror yang terjadi saat dilakukan eksekusi pada suatu operasi dengan type object yang tidak sesuai.
ZeroDivision eror merupakan eror yang terjadi saat eksekusi program menghasilkan perhitungan matematika dengan angka 0

\item Try except
cara memakai try except adalah sebagai berikut
    \lstinputlisting[firstline=8,lastline=17]{src/2/1144124/1144124_Teori.py}
    \end{enumerate}

\section{Tomy Prawoto}
\subsection{Soal 1}
Isi jawaban soal ke-1

Kalau mau dibikin paragrap \textbf{cukup enter aja}, tidak usah pakai \verb|par| dsb

%\subsection{Soal 2}
%Isi jawaban soal ke-2

%\subsection{Soal 3}
%Isi jawaban soal ke-3
